\documentclass{article}
\usepackage{graphicx} % Required for inserting images
\usepackage{amsmath}
\usepackage [separate-uncertainty] {siunitx}

\title{Semester Project--Double Pendulum \\Computational Physics Fall 2023 \\Draft 1}
\author{\textit{Robert Zhao, Jorge Gomez, Daniel Tanagho}}
\date{October 26, 2023}

\begin{document}

\maketitle

\section*{Introduction:}
The purpose of this project is to examine the behavior of a double pendulum. This system appears simple at first glance, but the related equations of motion are complex enough that makes it impossible to solve analytically. However, it is possible to find a numerical solution using various integration techniques, which will help us visualize the behavior of this system and examine the motion.

\section{Equations of Motion:}
We'll start by deriving the equations of motion. The system is composed of two pendulum bobs of equal mass $m$. Bob 1 is attached by a string of length $l$ to a pivot point. Bob 2 is attached to Bob 1 by a string of equal length $l$. The easiest way to find the equations of motion is to use Lagrangian Dynamics. Such that the Lagrange Equation is (where $T$ is the Kinetic Energy, and $U$ is the Potential Energy):
\begin{equation}
    \mathcal{L} = T - U
\end{equation}
For this, we will need to identify the energy associated with each pendulum bob. For simplicity, we will set the zero-point of potential energy for each bob at its equilibrium point. And we will identify the $+\hat{z}$ direction as the direction of \textit{decreasing} potential energy. We can do so with impunity, so long as we respect the same coordinatization throughout our calculations. We find:
\begin{align*}
\begin{split}
    U_1 & = -mgh_1 = -mgl\cos(\theta_1) \\
    U_2 & = -mgh_2 = -mgl[\cos(\theta_1)+\cos(\theta_2)]
\end{split}
\end{align*}
The Kinetic energy $K = mv^2/2$ is a bit trickier to find. But starting from writing the coordinates in the $x-z$ plane for each bob makes it much easier:
\begin{align*}
\begin{split}
    x_1 & = l\sin(\theta_1) \\
    x_2 & = l[\sin(\theta_1)+\sin(\theta_2)] \\
    z_1 & = l\cos(\theta_1) \\
    z_2 & = l[\cos(\theta_1)+\cos(\theta_2)]
\end{split}
\end{align*}
Then we take the time derivative:
\begin{align*}
\begin{split}
    \dot{x_1} & = l\cos(\theta_1)\dot{\theta_1} \\
    \dot{x_2} & = l[\cos(\theta_1)\dot{\theta_1} + \cos(\theta_2)\dot{\theta_2}] \\
    \dot{z_1} & = -l\sin(\theta_1)\dot{\theta_1} \\
    \dot{z_2} & = -l[\sin(\theta_1)\dot{\theta_1} + \sin(\theta_2)\dot{\theta_2}]
\end{split}
\end{align*}
Since $v_i^2 = \dot{x_i}^2 + \dot{z_i}^2$, then:
\begin{align*}
\begin{split}
    v_1^2 & = l^2\dot{\theta_1}^2 \\
    v_2^2 & = l^2[\dot{\theta_1}^2 + \dot{\theta_2}^2 + 2\dot{\theta_1}\dot{\theta_2}[\cos(\theta_1)\cos(\theta_2) + \sin(\theta_1)\sin(\theta_2)]]
\end{split}
\end{align*}
Using a useful trigonometric identity:
\begin{equation*}
    \cos(\alpha - \beta) = \cos(\alpha)\cos(\beta) + \sin(\alpha)\sin(\beta)
\end{equation*}
We obtain:
\begin{equation*}
        v_2^2 = l^2[\dot{\theta_1}^2 + \dot{\theta_2}^2 + 2\dot{\theta_1}\dot{\theta_2}\cos(\theta_1 - \theta_2)]
\end{equation*}
\\Then:
\begin{align*}
\begin{split}
    T_1 & = \frac{1}{2}ml^2\dot{\theta_1}^2 \\
    T_2 & = \frac{1}{2}ml^2[\dot{\theta_1}^2 + \dot{\theta_2}^2 + 2\dot{\theta_1}\dot{\theta_2}\cos(\theta_1 - \theta_2)]
\end{split}
\end{align*}
Since:
\begin{align*}
    \mathcal{L} = T_1 + T_2 - U_1 - U_2
\end{align*}
Then the Lagrangian of the system is:
\begin{equation}
    \mathcal{L} = ml^2[\dot{\theta_1}^2 + \frac{1}{2}\dot{\theta_2}^2 + \dot{\theta_1}\dot{\theta_2}\cos(\theta_1-\theta_2)] + mgl(2\cos(\theta_1) + \cos(\theta_2))
\end{equation}
\\Now that the hard part is done, we can use the Euler-Lagrange equation to find the equations of motion:
\begin{align*}
\begin{split}
    \frac{d}{dt}(\frac{\partial\mathcal{L}}{\partial\dot{\theta_1}}) & = \frac{\partial\mathcal{L}}{\partial\theta_1} \\ 
        \frac{d}{dt}(\frac{\partial\mathcal{L}}{\partial\dot{\theta_2}}) & = \frac{\partial\mathcal{L}}{\partial\theta_2}
\end{split}
\end{align*}
After differentiating and rearranging, we finally obtain:
\begin{equation}
    2\Ddot{\theta_1} + \Ddot{\theta_2}\cos(\theta_1 - \theta_2) + \dot{\theta_2}^2\sin(\theta_1-\theta_2) + 2\frac{g}{l}\sin(\theta_1) = 0
\end{equation}
\begin{equation}
   \Ddot{\theta_2} + \Ddot{\theta_1}\cos(\theta_1 - \theta_2) - \dot{\theta_1}^2\sin(\theta_1-\theta_2) + \frac{g}{l}\sin(\theta_2) = 0
\end{equation}
\\We notice that equations (3) and (4) are second-order, non-linear, ordinary differential equations. In the small-angle approximation, we can find analytic solutions, but as they stand now, we can only solve them numerically. So, we will define the angular frequency $\omega_i = \dot{\theta_i}$ in order to re-write these equations in first-order:
\begin{equation}
    2\dot{\omega}_1 + \dot{\omega}_2\cos(\theta_1 - \theta_2) + \omega_2^2\sin(\theta_1-\theta_2) + 2\frac{g}{l}\sin(\theta_1) = 0
\end{equation}
\begin{equation}
   \dot{\omega}_2 + \dot{\omega}_1\cos(\theta_1 - \theta_2) - \omega_1^2\sin(\theta_1-\theta_2) + \frac{g}{l}\sin(\theta_2) = 0
\end{equation}

\section{Total Energy and Stationary Points:}
There's a quantity that is important to keep track of during our simulation, and that is the total energy of the system, or the Hamiltonian.
\begin{equation}
    \mathcal{H} = T + U = ml^2[\omega_1^2 + \frac{1}{2}\omega_2^2 + \omega_1\omega_2\cos(\theta_1-\theta_2)] + mgl(2\cos(\theta_1) + \cos(\theta_2))
\end{equation}

\end{document}
